\documentclass{article}

\usepackage{graphicx}
\usepackage{amsmath}
\graphicspath{ {./images/} }

\usepackage[greek,english]{babel}
\usepackage{alphabeta}



\title{Άσκηση 6}
\author{Χρήστος Αλέξανδρος Τσιγγιρόπουλος}
\date{30 January 2022}


\begin{document}
    \maketitle
    Στην άσκηση αυτή υλοποιούνται 2 προσεγγίσεις του ολοκληρώματος του
    ημιτόνου με τις μεθόδους του Simpson και του Τραπεζίου στο διάστημα
    $[0,\frac{\pi}{2}$]. 
    Για τις προσεγγίσεις αυτές χρησιμοποιήθηκαν τα 11 σημεία :

    \begin{equation*}
        [0, \frac{\pi}{20}, \frac{\pi}{10},...,\frac{9\pi}{20}, \frac{\pi}{2}]
    \end{equation*}
    που δημιουργούν N=10 διαστήματα. Αποτέλεσμα του κώδικα είναι η προσέγγιση
    της κάθε μεθόδου ακολουθούμενη απο το αριθμητικό και το θεωρητικό σφάλμα.

    \section{Μέθοδος Τραπεζίου}
    Η προσέγγιση αυτή υπολογίζεται με τον τύπο :
    \begin{equation*}
        \frac{b-a}{2N}(f(x_0) + f(x_n) + 2 \sum_{i=1}^{n-1} f(x_i))
    \end{equation*}
    με a,b την αρχή και το τέλος του διαστήματος.
    Το αριθμητικό σφάλμα προκύπτει απο την απόλυτη αφαίρεση του πραγματικού 
    ολοκληρώματος ($cos(\frac{\pi}{2})$) με την προσέγγιση της μεθόδου και είναι 0.002. 
    Το θεωρητικό σφάλμα προκύπτει απο τον τύπο : 
    $\frac{(b-a)^3}{12N^2} M$ με 
    $M=max{|f''(x)|}$ δλδ στο παράδειγμα 
    μας $M=|-sin(\frac{\pi}{2})| = 1$ και το θεωρητικό σφάλμα βγαίνει 
    0.003. Αυτό σημαίνει ότι το αριθμητικό σφάλμα θα είναι μικρότερο απο αυτό όπου και ισχύει.
    
    \section{Simpson}
    Η προσέγγιση αυτή υπολογίζεται με τον τύπο :
    \begin{equation*}
        \frac{b-a}{3N}(f(x_0) + f(x_N) + 4 \sum_{i=1}^{\frac{n}{2}} f(x_{2i-1}) + 2 \sum_{i=1}^{\frac{n}{2}-1} f(x_{2i}))
    \end{equation*}
    Το αριθμητικό σφάλμα προκύπτει με τον ίδιο τρόπο και βγαίνει 0.000003.
    Το θεωρητικό σφάλμα προκύπτει απο τον τύπο : 
    $\frac{(b-a)^5}{180N^4} M$ με 
    $M=max{|f^{(4)}(x)|}$ δλδ στο παράδειγμα 
    μας $M=|sin(\frac{\pi}{2})| = 1$ και βγαίνει 
    0.000005. Αυτό σημαίνει ότι το αριθμητικό σφάλμα θα είναι μικρότερο απο αυτό όπου και ισχύει.
    
    
    \section{Συμπέρασμα}
    Το συμπέρασμα της λύσης είναι ότι η μέθοδος Simpson είναι πιο ακριβής σε σχέση 
    με την μέθοδο του τραπεζίου για το ίδιο διάστημα, αφού το σφάλμα είναι αρκετά μικρότερο για την πρώτη.
    

    
    
\end{document}