\documentclass{article}

\usepackage{graphicx}
\usepackage{amsmath}
\graphicspath{ {./images/} }

\usepackage[greek,english]{babel}
\usepackage{alphabeta}



\title{Άσκηση 3}
\author{Χρήστος Αλέξανδρος Τσιγγιρόπουλος}
\date{29 December 2021}


\begin{document}
    \maketitle
    \section{Άσκηση 3.1}
    
    Το πρόγραμμα ζητάει απο τον χρήστη το μέγεθος του πίνακα Α,
    και περιμένει μετά σαν είσοδο τον επαυξήμενο πίνακα. 
    Δλδ αν εισαχθεί στην αρχή το 3 θα περιμένει 4 στοιχεία (3 του πίνακα Α και 1 του πίνακα Β) +(αλλαγή γραμμής), 4 στοιχεία +(αλλαγή γραμμής) και άλλα 4 στοιχεία +(αλλαγή γραμμής).
    Άν όλα πάνε καλά το πρόγραμμα καλεί την συνάρτηση pa\_lu(Α,b) 
    που επιστρέφει 5 συναρτήσεις (P,A,L,U,Z) όπου PA=LU , Ux=y με Ax=b
    και Ly=Pb=z . Στην συνέχεια καλώ την συνάρτηση ly\_z(L,Z) που επιστρέφει τον πίνακα y. Τέλος, καλείται η συνάρτηση ux\_y(U, y)
    που επιστρέφει τον πίνακα x δλδ τον πίνακα με τις ρίζες που τις τυπώνει.
    
    Ειδικότερα για τις 3 συναρτήσεις:
    \begin{itemize}
        \item pa\_lu(Α,b)
        
        Αρχικά αρχικοποιούμε τους πίνακες p,u,l στον p (0 παντού και 1 στην διαγώνιο), u=A ,στον  l (0 παντού).
        Στην συνέχεια κάνω οδήγηση στους πίνακες u,b,l,p με βάση τον u για το [k,k] στοιχείο του u.
        Έπειτα βάζω στον l τον σωστό συντελεστή και στον u κάνω gauss για να δημιουργήσω τον άνω τριγωνικό της τριγωνιοποίησης. 
        Τέλος βάζω 1 στην κύρια διαγώνιο του πίνακα l για να πάρω τον τελικό πίνακα.
        Στο τέλος ο πίνακας Z είναι ίσος με τον αλλαγμένο b. 	
        Επιστρέφω τους 5 πίνακες (p,a,l,u,b)  με a τον πίνακα που μπήκε σαν όρισμα.
        
        \item ly\_z(L,Z)
        
        Η συνάρτηση αυτή λύνει το Ly=Z ως προς y και επιστρέφει τον πίνακα y.
        Ειδικότερα, αρχικοποιεί τον y με 0 και για κάθε i τοποθετεί σε αυτό την λύση. Η λύση για κάθε i είναι 
        y[i] = (z[i]-sum)/L[i][i] με 
        \begin{equation*}
            sum = \sum_{j=1}^{i-1}L[i][j] \cdot y[j] + \sum_{j=i+1}^{n}L[i][j] \cdot y[j]
        \end{equation*}

        
        \item ux\_y(U,y)

        Η συνάρτηση αυτή λύνει το Ux=y ώς προς x δλδ επιστρέφει τον πίνακα x, που περιέχει τις ρίζες του
        αρχικού προβλήματος Ax=b. Κάνει το ίδιο με την πάνω συνάρτηση με την διαφορά ότι το i ξεκινάει απο 
        το τέλος προς στην αρχή γιατί ο πίνακας U είναι άνω τριγωνικός. Διαφορετικά δεν θα μπορούσε να βγεί 
        το αποτέλεσμα. Με αυτόν τον τρόπο η πρώτη ρίζα που βρίσκουμε είναι η xn μετά η xn-1 και στο τέλος την x1. Η λύση για κάθε i σε αυτή την περίπτωση είναι 
        x[i] = (y[i]-sum)/U[i][i] με 
        \begin{equation*}
            sum = \sum_{j=n}^{i+1}U[i][j] \cdot x[j] + \sum_{j=i-1}^{0}U[i][j] \cdot x[j]
        \end{equation*}


    Παράδειγμα: για είσοδο: 3
    
    2 1 5 5

    4 4 -4 0

    1 3 1 6

    Το πρόγραμμα τυπώνει :
    
    x1 = -1.000 
    x2 = 2.000 
    x3 = 1.000 
    \end{itemize}
    
    \section{Άσκηση 3.2}
    
    Το πρόγραμμα έχει ορισμένο ένα πίνακα Α και τυπώνει τον κάτω τριγωνικό πίνακα απο την αποσύνθεση Cholesky.
    Ειδικότερα, αρχικοποιεί ένα πίνακα L (nxn) με 0 με n = μέγεθος(Α). Στην συνέχεια αποθηκεύεται : 
    \begin{equation*}
        L[k][i] = \frac{A[k][i] - \sum_{j=1}^{i-1}L[i][j] \cdot L[k][j]}{L[i][i]} , k\neq i
    \end{equation*}
    \begin{equation*}
        L[k][k] = \frac{A[k][k] - \sum_{j=1}^{k-1} L^2[k][j]}{L[i][i]} , k=i 
    \end{equation*}
    
    και στο τέλος τυπώνει τον πίνακα L.
    
    \section{Άσκηση 3.3}
    
    Στό πρόγραμμα αυτό, αρχικοποιούμε με τις συναρτήσεις initialize\_a(n) και initialize\_b(n)
    τέσσερις πίνακες, τους a10,b10 με n=10 και a10000,b10000 με n=10000.
    Στην συνέχεια, καλούμε την συνάρτηση gauss\_seidel(a,b), αρχικά για τους πίνακες a10,b10.
    Η συνάρτηση αυτή επιστρέφει έναν πίνακα x που περιέχει τις προσεγγίσεις των ριζών.
    
    Πιο συγκεκριμένα η συνάρτηση αυτή , αρχικοποιεί δύο νέους πίνακες x0,xn (nxn) με 0 . Έπειτα για κάθε διαφορετικό i τοποθετεί στο xn[i] : 
    \begin{equation}
        xn[i] = \frac{b[i]-\sum_{j=1}^{i-1}a[i][j]\cdot \textbf{x0[i]} - \sum_{j=i+1}^{n}a[i][j]\cdot \textbf{xn[i]}}{a[i][i]}
    \end{equation}
    και αποθηκεύουμε τον νέο πίνακα xn στον πίνακα x0.
    
    Aφού γίνουν αυτά καλούμε την συνάρτηση find(a,b,xn) που επιστρέφει True/False 
    δλδ χρησιμοποιείτε για τον έλεγχο τερματισμού. 
    Σε αυτή, ελέγχουμε αν η άπειρη νόρμα στο σφάλμα της λύσης, είναι μικρότερη απο το 0.5e-4 δλδ : 
    \begin{equation*}
        \big\|A\big\|_\infty \leq 0.5\cdot10^{-4}
    \end{equation*}
    Αν είναι μεγαλύτερη τότε γυρνάει True και επαναλαμβάνουμε από την (1). Αν είναι μικρότερη τότε γυρνάει False και τελειώνει η συνάρτηση, τυ\-πώ\-νο\-ντας τις επαναλήψεις που χρειάστηκαν και επιστρέφοντας την λίστα με τις προσεγγίσεις.
    
    Για τους πίνακες a10,b10 εμφανίζονται και οι προσεγγίσεις, ενώ για τους πίνακες a10000,b10000 εμφανίζονται μόνο οι επαναλήψεις.
    
\end{document}