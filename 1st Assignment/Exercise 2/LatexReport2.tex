\documentclass{article}

\usepackage{graphicx}
\usepackage{amsmath}
\graphicspath{ {./images/} }

\usepackage[greek,english]{babel}
\usepackage{alphabeta}



\title{Άσκηση 2}
\author{Χρήστος Αλέξανδρος Τσιγγιρόπουλος}
\date{29 December 2021}

\begin{document}
    \maketitle
    \section{Αποτελέσματα}
        
    Μέθοδος Διχοτόμησης :
    
    [-2,-1]  Η ρίζα είναι : -1.3812984975  και χρειάστηκαν 23  επαναλήψεις
    
    [-1,0]  Η ρίζα είναι : -0.6668368474  και χρειάστηκαν 1175  επαναλήψεις
    
    [0,0.25]  Η ρίζα είναι : 0.2051828635  και χρειάστηκαν 27  επαναλήψεις
    
    [0.25,1]  Η ρίζα είναι : 0.4999998792  και χρειάστηκαν 30  επαναλήψεις
    
    [1,2]  Η ρίζα είναι : 1.1761155447  και χρειάστηκαν 31  επαναλήψεις
 
    Newton - Raphson    :
    
    [-2,-1]  Η ρίζα είναι : -1.3812984884  και χρειάστηκαν 4  επαναλήψεις
    
    [-1,0]  Η ρίζα είναι : -0.6668720140  και χρειάστηκαν 7  επαναλήψεις
    
    [0,0.25]  Η ρίζα είναι : 0.2051829413  και χρειάστηκαν 2  επαναλήψεις
    
    [0.25,0.75]  Η ρίζα είναι : 0.5000000646  και χρειάστηκαν 3  επαναλήψεις
    
    [1,2]  Η ρίζα είναι : 1.1761155574  και χρειάστηκαν 5  επαναλήψεις
    
    Μέθοδος Τέμνουσας   :
    
    [-1.5,-1.2]  Η ρίζα είναι : -1.3812984820  και χρειάστηκαν 6  επαναλήψεις
    
    [-1,0]  Η ρίζα είναι : -0.6668840272  και χρειάστηκαν 16  επαναλήψεις
    
    [0,0.25]  Η ρίζα είναι : 0.2051829247  και χρειάστηκαν 4  επαναλήψεις
    
    [0.25,0.8]  Η ρίζα είναι : 0.4999999988  και χρειάστηκαν 6  επαναλήψεις
    
    [1,2]  Η ρίζα είναι : 1.1761155574  και χρειάστηκαν 9  επαναλήψεις
    
    
    \section{Αλγόριθμος Διχοτόμησης με random()}
    
    Αν τρέξουμε τον αλγόριθμο αυτό θα παρατηρήσουμε ότι δεν συκλίνει πάντα σε ίδιο αριθμό επαναλήψεων.
    Αυτό συμβαίνει γιατί το νέο «μέσο» βασίζεται πάνω στην συνάρτηση random() που κάθε φορά επιστρέφει διαφορετικό αριθμό.
    Ειδικότερα για την διπλή ρίζα (0.666..) κάποιες φορές χρείαζεται γύρω στις 250 και κάποιες άλλες φορές μπορεί να ξεπεράσει και τις 1500 επαναλήψεις. 
    Αυτό συμβαίνει επειδή στο διάστημα [-1,0] κοντά στην διπλή ρίζα 
    δεν ισχύει το θεώρημα Bolzano αφού έχουμε: 
    \begin{equation}
    f(x) , f(rand) , f(y) < 0
    \end{equation} 
    οπότε η συνάρτηση θα βγάλει αποτέλεσμα μόνο όταν το f(rand) είναι πολύ κοντά στο 0 δλδ όταν η τιμή rand είναι πολύ κοντά στην ρίζα αφού τα άκρα του διαστήματος δεν θα αλλάξουν ποτέ.
    
    
    \section{Σχόλια Αποτελέσματος-Σύγκριση με τις κλασικές μεθόδους}
    
    Το πολυώνυμο αυτό έχει 5 ρίζες την (-1.81298), την (-0.666...)  που είναι διπλή, την (0.20518), την (0.5) και την (1.1761).
    Η αλλαγμένη Newton-Raphson μέθοδος είναι αρκετά πιο γρήγορη απο την παλιά,
    με μέσο όρο 4,2 επαναλλήψεις για κάθε ρίζα έναντι 6,8 της κλασικής. 
    Η τροποποιημένη μέθοδος τέμνουσας είναι και αυτή επίσης λίγο πιο γρήγορη με μέσο όρο 8,2 έναντι 10 επαναλλήψεων απο την κλασική. 
    Τέλος για την μέθοδο διχοτόμησης τα αποτελέσματα δέν είναι σταθερά καθώς η αλλαγμένη μέθοδος χρησιμοποιεί την συνάρτηση rand() αλλά είναι σίγουρα καλύτερη, καθώς για την ρίζα (-0.666...) η κλασική δεν μπορεί να την υπολογίσει, αφού είναι διπλή και δεν ισχύει το θεώρημα Bolzano.
    
    (Οι μετρήσεις έγιναν για τα ίδια ακριβώς διαστήματα και με τις συναρτήσεις της άσκησης 1.)
    
    
    \section{Περιγραφή Κώδικα}
    
    Ίδιος με της προηγούμενης άσκησης με τις διαφορές στις συναρτήσεις όπως ορίστηκαν απο την άσκηση.
    Αξίζει να σημειωθεί ότι για την διχοτόμηση το νέο «μέσο» είναι ίσο με το γινόμενο της διαφορας του προηγούμενου διαστήματος (y-x) επί την τιμή που επιστρέφει η συνάρτηση random()=(0,1) σύν το χ. 
    Αυτό γιατί το γινόμενο μας γυρνάει μια τιμή μεταξύ του [0,y-x] και με την πρόσθεση του χ έχουμε μία τυχαία τιμή μεταξύ του [x,y]. 
    
    
\begin{figure}[t]
    \section{Γραφική παράσταση}
    \includegraphics[width=12cm]{two.png}
\end{figure}
\end{document}